\section{Point Estimation of the Proportion of COVID-19 \\ Positive Cases in the Survey}

\subsection{Point Estimation}

The following report presents a \textbf{point estimation analysis} regarding the proportion of individuals tested positive for COVID-19 within our survey. The aim is to provide a precise estimate of the mean percentage of individuals who tested positive for COVID-19 based on the collected survey data of $230$ individuals.

\

Out of $230$ participants, $107$ individuals have been identified as having undergone COVID-19 tests. The point estimation we want to compute is based on the proportion of positive results within the surveyed population.

\ 

Consider $X_{1}, \ \ldots \ , X_{n}$ being the random variables representing COVID-19 test results for $n$-individuals.

\ 

Now, let $p$ be the probability that a particular person has a positive test result, i.e., $X_{i}$ being "Yes" or $1$ has the probability $p$.

\ 

Then,

\[ 
X_{i} = \left\{
\begin{array}{ll}
        1 & \text{with probability } \textit{p} \text{ (indicates "\textbf{Yes}" instance)} \\
        0 & \text{with probability } \textit{1 - p} \text{ (indicates "\textbf{No}" instance)}
\end{array} 
\right. 
\]
$$\forall \text{ \ i \ = \ 1 \ \ldots \, n}.$$

and each individual should have their independent test result. Thus :
$$X_{1} \ \ldots \ X_{n} \ \sim \ \text{IID Ber(\textbf{p})}$$
Therefore, $E(X_{i}) \ = \ p, \ \ Var(X_{i}) \ = \ p(1 - p) \ \ \forall \ i \ = \ 1 \ \ldots \ \ldots \ n$.

\

Here, a natural choice to estimate the population mean $p$ is the sample mean
$$\Bar{x} \ = \ \frac{x_{1} + \ldots + x_{n}}{n}$$
which is our test-statistic $T$.
$$T(X_{1}, \ \ldots \ , X_{n}) \ = \ \Bar{X} \ = \ \frac{X_{1} + \ldots + X_{n}}{n}$$

\ 

Now,
$$E[T(X_{1}, \ \ldots \ , X_{n})] \ = \ E[\Bar{X}] \ = \ \frac{1}{n} \cdot \{ E[X_{1}] + \ \ldots \ + E[X_{n}] \}$$
$$= \ \frac{1}{n}\cdot n\cdot p \ = \ p \ \ \text{(as $E[X_{i}] \ = \ p) \ \ \forall \ i \ = \ 1,\ 2, \ \ldots \ ,\ n$ } $$

\ 

Hence, $T(X_{1}, \ \ldots \ , X_{n}) \ = \ \Bar{X}$ is an unbiased estimator of $p$.

\ 

In our observed data, $n \ = \ 107$
$$\text{Sample mean} \ = \ \Bar{x} \ = \ \frac{x_{1} + \ \ldots \ + x_{n}}{n} \ = \ \frac{48}{107} \ \approx \ 0.448598$$
Since, the number of COVID positive results in the data is $48$ out of $107$ positive tests.

\

Sample Variance, $SD \ = \ \sqrt{\frac{48 \ \times \ 59}{106 \ \times \ 107}} \ \approx \ 0.49691$ \\
Thus, sample mean $\Bar{x} \ = \ 0.448598$ being an estimate of $p$ while $\Bar{X}$ being estimator for population quantity $p$ along with standard error,
$$SE \ = \ \frac{SD}{\sqrt{n}} \ = \ 0.048307$$
Since, $SE$ here is very small, therefore for large sample sizes (i.e. when $n \ \to \ \infty$), the standard error decreases further. 

\ 

\subsection{Maximum Likelihood Estimate (MLE)}

Now, we follow the following steps to compute the MLE of $p$.

$$L(p) \ = \ \prod_{i = 1}^{n} \Pr \{ X_{i} \} \ = \ \prod_{i = 1}^{n} p^{x_{i}} \cdot (1  - p)^{1 - x_{i}} \ = \ p^{\sum_{i = 1}^{n} x_{i}} \cdot (1 - p)^{n \ - \ \sum_{i = 1}^{n} x_{i}}$$
where, $$\Pr \{ x_{i} \} \ = \ P[X_{i} \ = \ x_{i}] \ = \ p^{x_{i}} \cdot (1-p)^{1 \ - \ x_{i}}$$

\ 

Here, we will maximize the likelihood $l(p) \ = \ \log_{e} L(p)$
$$l(p) \ = \ \sum_{i = 1}^{n} x_{i} \cdot \log p \ + \ (n \ - \ \sum_{i = 1}^{n} x_{i}) \cdot \log (1 - p)$$
$$l^{\prime} (p) \ = \ \frac{\sum_{i = 1}^{n} x_{i}}{p} \ - \ \frac{n - \sum_{i = 1}^{n} x_{i}}{1 - p}$$

$$ \text{For } l^{\prime} (p) \ = \ 0 \ \implies \ (1 - p) \cdot \sum_{i = 1}^{n} x_{i} \ = \ p \cdot (n \ - \ \sum_{i = 1}^{n} x_{i})  $$

$$\implies \ p \ = \ \frac{\sum_{i = 1}^{n} x_{i}}{n} \ = \ \Bar{x}$$

We have,
$$l^{\prime \prime}(p) \vert_{p = \Bar{x}} \ = \ \left(- \ \frac{\sum_{i = 1}^{n} x_{i}}{p^{2}} \ - \ \frac{n - \sum_{i = 1}^{n} x_{i}}{(1 - p)^{2}}\right) \Big|_{p = \bar{x}} \ = \ - \ \frac{n}{\Bar{x}} \ - \ \frac{n \cdot (1 - \Bar{x})}{(1 - \Bar{x})^{2}} \ = \ - \ \frac{n}{\Bar{x}} \ - \ \frac{n}{1 - \Bar{x}}$$

\ 

Substituting our observed value of $n, \ \Bar{x}$ we get,
$$l^{\prime \prime}(p) \vert_{p = \Bar{x}} \ = \ - \frac{n}{\Bar{x}(1 - \Bar{x})} \ = \ - \ 481.54 \ < \ 0 \ \ \forall \ p $$

Thus, we have $l(p)$ is maximum at $p \ = \ \Bar{x}$. So finally,
$$\hat{p} \ = \ \Bar{x} \ = \ \frac{x_{1} + \ \ldots \ + x_{n}}{n}$$

From our observed data,
$$\hat{p} \ = \ 0.448598$$ 
is the \textbf{MLE} of $p$.

\ 

Hence, the MLE indicates that approximately $44.8598 \%$ of the surveyed \\ population tested positive for COVID-19. This estimation is an alignment with the earlier point estimate derived using sample proportions.

\

Thus, $\Bar{X}$ being consistent estimator of $p$ as well as unbiased.

\ 

While the MLE`s are the consistent estimator and provide a robust estimate based on the given sample data, further investigation considering larger sample sizes (i.e. when $n \to \infty$) could refine the accuracy of the estimation.

\ 

\section{Interval Estimation of Mean Difference of Monthly Expenditure \\ (Before-After COVID)}

Here, we want to estimate the mean difference of monthly medical expenditure. We have collected data from $230$ individuals, depending on which we carry out the estimation : \\

Consider $(X_{1}, \ Y_{1}) \ , \ \ldots \ ,  \ (X_{n}, \ Y_{n})$ are paired data on $n$-individuals where
$$\{ X_{1}, \ \ldots \, X_{n}\} : \ \text{be the monthly expenditure of $n$-individuals before COVID}$$
$$\{ Y_{1}, \ \ldots \, Y_{n}\} : \ \text{be the monthly expenditure of $n$-individuals after COVID}$$

\ 

Let's Take, 
$$D_{i} \ = \ Y_{i} \ - \ X_{i} , \ \ \forall \ i \ = \ 1, 2,\ \ldots, \ n$$
$$\Bar{D} \ = \ \frac{1}{n} \cdot \sum_{i = 1}^{n} D_{i} \ \sim \ N (\mu_{d}, \sigma_{d}^{2})$$
where $\mu_{d} \ = \ \mu_{y} \ - \ \mu_{x}$ and $\mu_{x}$, \ $\mu_{y}$ are the population means of $X$ and $Y$ respectively.

\ 

We want to estimate $\mu_{d}$ by the \textit{Sample mean difference} $\Bar{d}$ i.e. $\hat{\mu_{d}} \ = \ \Bar{d}$.

\ 

As, $E(\Bar{d}) \ = \ \frac{1}{n} \cdot \sum_{i = 1}^{n} E(d_{i}) \ = \ \frac{n \cdot \mu_{d}}{n} \ = \ \mu_{d}$, so $\Bar{d}$ is an unbiased estimator of $\mu_{d}$.\\

Also, $\frac{\Bar{D} - \mu_{d}}{\frac{s_{d}}{\sqrt{n}}} \ \sim \ t_{n - 1}$, where $s_{d}$ is the \textit{Sample standard deviation}.

\

Then the $95\%$ paired-$t$ confidence interval for $\mu_{d}$ is given by 
$$\mu_{d} \ \pm \ t_{(0.025, \ n-1)} \ \times \ \frac{s_{d}}{\sqrt{n}}$$
So, from the collected data, we have : \\
\fbox{\parbox{\textwidth}{
$$\Bar{d} \ = \ 1920.739$$
$$s_{d} \ = \ 9756.072$$
$$n \ = \ 230$$
$$\text{Confidence} \ = \ 0.95$$
$$t_{0.025, 229} \ = \ 1.970377$$
\begin{center}
So, Confident interval for $\mu_{d} = (653.2025, \ 3188.275)$
\end{center}
}}

\ 

\textbf{Conclusion:} We have a $95\%$ confidence that the mean difference of monthly medical expenditure falls within the interval $(653.2025, \ 3188.275)$. This suggests that, on average, monthly expenses have increased by significant margins during the COVID period.